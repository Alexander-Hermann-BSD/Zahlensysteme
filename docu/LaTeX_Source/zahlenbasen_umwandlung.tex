\documentclass[a4paper,12pt,twoside]{book}
\usepackage{ucs}
\usepackage{a4wide}
\usepackage[utf8x]{inputenc}
\usepackage[german]{babel}
\usepackage[pdftex]{graphicx}
\usepackage[pdftex]{hyperref} 
\usepackage{graphfig}
\usepackage{makeidx}
\usepackage{pdfpages}
\usepackage{amsmath}
\usepackage[german]{struktex}
\usepackage{color}
\usepackage{listings}

\bibliographystyle{geralpha}

\hypersetup{ 
	pdfdisplaydoctitle=true,
	pdfpagelayout=TwoPageRight,
	pdfinfo={
		Title={Zahlenbasen Umwandlung – in Theorie und Praxis}, 
		Author={Alexander Hermann},
		Subject={Zahlenbasen Umwandlung}, 
		Webseite={http://demo.hermann-bsd.de/zahlenbasen/},
		% ...
	} 
}

% \sProofOn
\definecolor{middlegray}{rgb}{0.5,0.5,0.5}
\definecolor{lightgray}{rgb}{0.8,0.8,0.8}
\definecolor{orange}{rgb}{0.8,0.3,0.3}
\definecolor{yac}{rgb}{0.6,0.6,0.1}
\definecolor{darkgreen}{rgb}{0.0,0.6,0.0}
 
\lstset{
   basicstyle=\scriptsize\ttfamily,
   keywordstyle=\bfseries\ttfamily\color{orange},
   stringstyle=\color{darkgreen}\ttfamily,
   commentstyle=\color{middlegray}\ttfamily,
   emph={square}, 
   emphstyle=\color{blue}\texttt,
   emph={[2]root,base},
   emphstyle={[2]\color{yac}\texttt},
   showstringspaces=false,
   flexiblecolumns=false,
   tabsize=2,
   numbers=left,
   numberstyle=\tiny,
   numberblanklines=false,
   stepnumber=1,
   numbersep=10pt,
   xleftmargin=15pt,
   breaklines
}

\title{Zahlenbasen Umwandlung -- in Theorie und Praxis}
\author{Alexander Hermann}

\makeindex
\begin{document}
\maketitle
\tableofcontents
\lstlistoflistings 
\listoffigures

% \abstract{Umwandlung von Zahlenbasen der Basis $b_{1}$ in Basis $b_{2}$.}
\chapter{Einleitung}
Zahlen\index{Zahlen} in verschiedenen Zahlenbasen\index{Zahlen!-basen} werden im Wesentlichen für eine Vorvereinfachung zur menschlichen Kommunikation bzw.\ zur Umschreibung mit maschinellen Automatisierungen verwendet. 
Bei der Ausführung von Software\index{Software} auf reiner Hardware\index{Hardware}-Ebene läuft alles letztendlich rein binär\footnote{auf der Basis $b = 2$}\index{binär}\index{Binärsystem} ab. 

Da wir Menschen es gewohnt sind im Dezimalsystem\footnote{Basis $b = 10$} zu rechnen -- was möglicherweise daran liegt, dass der Mensch zehn Finger hat -- und auch dafür ausgebildet wurden, ist es im Allgemeinen einfacher, auf dieser Basis zu rechnen.
\section{Zahlendarstellung}
In vielen Programmiersprachen werden die Zahlensysteme\index{Zahlen!-systeme} \textbf{binär}\index{binär}, \textbf{octal}\index{octal}, \textbf{dezimal}\index{dezimal} und \textbf{hexadezimal}\index{hexadezimal} im Programmiercode zur Vereinfachung bzw.\ zur korrekten Interpretation durch den Compiler unterschiedlich eingegeben.
\begin{itemize}
	\item \textbf{binär:}\index{binär} \texttt{0b101110}
	\item \textbf{octal:}\index{octal} \texttt{0c576302}
	\item \textbf{dezimal:}\index{dezimal} \texttt{964}
	\item \textbf{hexadezimal:}\index{hexadezimal} \texttt{0xAFFE09}
\end{itemize}
Da hier aber generell alle möglichen Zahlensysteme verwendet werden, bzw.\ die verallgemeinerte Form der Umrechnung erklärt werden soll, werden im Folgenden Zahlen eines bestimmten Zahlensystems der Basis $b$ wie folgt dargestellt:
\[bxv\]
Wobei $b$ für die entsprechende Basis steht, $x$ zur Markierung immer als \textbf{x} verwendet wird und $v$ der Wert im entsprechenden Zahlensystem\index{Zahlen!-system} ist.
\subsection{Beispiel dazu}
Ähnlich wie oben:
\begin{itemize}
	\item \textbf{binär:}\index{binär} \texttt{2x101110}
	\item \textbf{octal:}\index{octal} \texttt{8x576302}
	\item \textbf{dezimal:}\index{dezimal} \texttt{10x964}
	\item \textbf{hexadezimal:}\index{hexadezimal} \texttt{16xAFFE09}
\end{itemize} 
Wäre das alles, wäre es wohl kaum nötig eine zusätzliche Darstellung zu verwenden.
Aber an eher seltenen Zahlensystemen\index{Zahlen!-system}, ist eine generalisierte Darstellung dann doch vorteilhaft:
\begin{itemize}
	\item \textbf{ternär:}\index{ternär} \texttt{3x211201}
	\item \textbf{quinär:}\index{quinär} \texttt{5x402314}
	\item \textbf{tridezimal}\index{tridezimal} \texttt{13x5A9C0B3}
	\item \textbf{oktovigesimal}\index{oktovigesimal} \texttt{28xNOR70KRANK}
	\item \textbf{hexatridezimal:}\index{hexatridezimal} \texttt{36xGIRAFFE0Z6A}
\end{itemize}
\section{Hier mögliche Zahlen}
Es gibt auch durchaus Umwandlungsmethoden, um $\real$eale Zahlen\index{Zahlen} umzurechnen.
Hier wird aber nur mit $\nat$atürlichen Zahlen gearbeitet.

\chapter{Umwandlungen}
\section{Umwandlung von Zahlen der Basis $b$ in das Zahlensystem der Basis 10}
\subsection{Allgemeine Formel zur Wandlung von Basis $b$ zu Basis 10}
Diese Formel ist für Zahlenbasen\index{Zahlen!-basen} der Basis $b=2$ bis Basis $b=36$\footnote{Basis 36 bei \texttt{ASCII}; bei \texttt{UTF-8} auch größer} mit den Ziffern\index{Ziffern} $0$ bis $9$ und den Buchstaben $A$ bis $Z$ möglich.
Die Formel setzt sich zusammen aus der Basis $b$, der Stellenposition\footnote{von rechts nach links} $s$ und dem angezeigten Wert $w$.
Wenn der Wert ein Buchstabe ist, ist der Wert gleich Buchstabenstelle $bu_{s}$ im Alphabet $ + 9$ \(w = bu_{s} + 9\) ansonsten der Zahlenwert\index{Zahlen!-wert} \(w = w\). 
Die Anzahl der maximalen Zeichen ist der Basiswert.\footnote{Deswegen ist 2 auch die kleinstmögliche Zahlenbasis, weil bei nur einem Zeichen kein Unterschied mehr möglich ist.} \marginpar{\textbf{Merke:} \textit{das erste Zeichen ist immer 0!}} 
Im "`normalen"', dezimalen\index{dezimal} Zahlensystem\index{Zahlen!-system} von $0$ bis $9$ ist die $10$ bereits \textit{zweistellig}.
\begin{equation}
x_{s} = b^{s - 1} * w
\label{eq:b-10}
\end{equation}
Die Ergebnisse der einzelnen Stellen werden summiert.

\subsection{Berechnungsablauf}
Der Berechnungsablauf kann wie in dem, in Abbildung~\ref{fig:struct1} dargestellten Struktogramm dargestellt werden.
\begin{figure}
	\begin{struktogramm}(161,185)	
	\assign{Umwandlung von Basis $b$ in Basis 10.}
	\assign%
	{
		\begin{declaration}[Parameter:]
			\description{\pVar{b}}{Eine \pKey{int} Variable, die die zu benutzende Zahlenbasis angibt.}
			\description{\pVar{q}}{Eine \pKey{string} Variable, die die zu übersetzende Zahl der Zahlenbasis \pVar{b} enthält.}	
		\end{declaration}
		\begin{declaration}[lokale Variablen:]
			\description{\pVar{step}}{Eine \pKey{int} Variable, die den aktuellen Schritt anzeigt.}
			\description{\pVar{result}}{Eine \pKey{int} Variable, die das Endergebnis beinhaltet.}
			\description{\pVar{charW}}{Eine \pKey{char} Variable für einen einzelnen Stellenwert des Eingabewerts \pVar{q}.}
			\description{\pVar{intW}}{Eine \pKey{int} Variable, die  den Integer-Wert des aktuellen Stellenwerts \pVar{charW} darstellt.}	
			\description{\pVar{laenge}}{Eine \pKey{int} Variable für die Länge des Quell-Strings \pVar{q}.}
			\description{\pVar{z}}{Eine \pKey{int} Variable als Zähler.}
		\end{declaration}
	}
	\sub{Länge von \pVar{q} abfragen}
	\return{\pVar{laenge} zurückgeben}
	\assign{\pVar{result}$ = 0$}
	\assign{\pVar{step} = \pVar{laenge}\( - 1\)}
	\assign{\pVar{z}$ = 0 $}
	\while[8]{\pVar{step}\( > 0\)}
		\sub{Character von \pVar{q} an Stelle \pVar{step} abfragen}
		\return{\pVar{charW} zurückgeben}
		\ifthenelse{5}{5}
			{\pVar{charW} ist eine Zahl}{\sTrue}{\sFalse}
			\assign[17]{\pVar{intW} $ = $ intwert(\pVar{charW})}
		\change	
			\sub{Wert des Characters ausrechnen}
			\return{\pVar{intW} zurückgeben}
		\ifend
		\assign{\pVar{step} $ = $ \pVar{step}\( - 1\)}
		\sub{Zwischenergebnis \pVar{intW} an Zähler \pVar{z} und Basis \pVar{b} berechnen.}
		\return{\pVar{intW} zurückgeben}
		\assign{\pVar{result}$ = $\pVar{result}$ + $\pVar{intW}}
		\assign{\pVar{z}$ = $\pVar{z}$ + 1$}
	\whileend 
	\assign{\pVar{result} zurückgeben}
\end{struktogramm}	
	\caption{Struktogramm Umwandlung in das Dezimalsystem}
	\label{fig:struct1}
\end{figure}
Die meisten Programmiersprachen haben vordefinierte Funktionen zur Längenberechnung von \pKey{string}-Variablen; ebenso gibt es Funktionen um an bestimmten Stellen eines Strings einzelne Zeichen abzurufen.
Damit entfällt die genauere Beschreibung der Längenabfrage und der Stellenabfrage.
Was hier noch fehlt, ist die Berechnung des Character-Werts, dies wird im Struktogramm in Abbildung~\ref{fig:struct1b} dargestellt, falls es sich \textbf{nicht} um eine Zahl handelt, so wie die Berechnung des Zwischenergebnisses, welches im Struktogramm in Abbildung~\ref{fig:struct1c} dargestellt wird.
\begin{figure}
	\begin{struktogramm}(135,62)
	\assign{Berechnung des Character-Werts}
	\assign%
	{
		\begin{declaration}[Parameter:]
			\description{\pVar{c}}{Eine \pKey{char} Variable, die den auszuwertenden Character angibt.}
		\end{declaration}
		\begin{declaration}[lokale Variablen:]
			\description{\pVar{result}}{Eine \pKey{int} Variable, die das Endergebnis beinhaltet.}
			\description{\pVar{z}}{Eine \pKey{int} Variable als Zwischenwert.}
		\end{declaration}
	}
	\assign{Lese den \texttt{ASCII} oder \texttt{UTF-8} Wert des Parameters \pVar{c} aus und schreibe es in das Zwischenergebnis \pVar{z}.} 
	\assign{\pVar{result}=\pVar{z} + 9 -- erste Buchstabenposition}
	\assign{\pVar{result} zurückgeben}
\end{struktogramm}

	\caption{Struktogramm Berechnung des Character-Werts}
	\label{fig:struct1b}
\end{figure}
\begin{figure}
	\begin{struktogramm}(161,83)
	\assign{Berechnung des Zwischenergebnisses}
	\assign%
	{
		\begin{declaration}[Parameter:]
			\description{\pVar{b}}{Eine \pKey{int} Variable, die die zu benutzende Zahlenbasis angibt.}
			\description{\pVar{w}}{Eine \pKey{int} Variable, die den Eingabewert angibt.}
			\description{\pVar{p}}{Eine \pKey{int} Variable, die die zu benutzende Stellenposition enthält.}	
		\end{declaration}
		\begin{declaration}[lokale Variablen:]
			\description{\pVar{z}}{Eine \pKey{int} Variable als Zähler.}
			\description{\pVar{result}}{Eine \pKey{int} Variable für das Endergebnis}
		\end{declaration}
	}
	\assign{\pVar{z}$=0$}
	\assign{\pVar{result}$=1$}
	\while[8]{\pVar{z}$<$\pVar{p}}
		\assign{\pVar{result}$ = $\pVar{result}$ * $\pVar{b}}
		\assign{\pVar{z}$=$\pVar{z}$+1$}
	\whileend
	\assign{\pVar{result}$=$\pVar{result}$*$\pVar{w}}
	\assign{\pVar{result} zurückgeben}
\end{struktogramm}

	\caption{Struktogramm Berechnung des Zwischenergebnisses}
	\label{fig:struct1c}
\end{figure}

\section{Umwandlung von Zahlen der Basis 10 in das Zahlensystem der Basis $b$}
Eine der Anleitungen fand ich im Web\footnote{http://www.arndt-bruenner.de \cite{zs:ab:2015}}.
Die umzurechnende Zahl\index{Zahl} $z$ wird durch die Basis\index{Basis} $b$ geteilt; der Quotient\index{Quotient} $q$ wird zur erneuten Rechnung verwendet; der jeweilige Rest\index{Rest} $r$ wird mit 10 hoch dem Rechenschritt $s$ multipliziert; der erste Rechenschritt ist $s = 0$. Es wird so häufig gerechnet, bis der Quotient 0 ist. 
\subsection{Allgemeine Formel zur Wandlung von Basis 10 zu Basis $b$}
In dieser Formel wird der Quotient des vorherigen Rechenschritts als das Zwischenergebnis $s_{n}$ bezeichnet, wobei $n$ die Nummer des Rechenschritts ist. Die Zählung der Rechenschritte fängt mit 0 an.
Also ist für die erste Stelle $s_{0}$ die umzuwandelnde Zahl $z$ der Basis $b$ zu verwenden.
\begin{equation}
\frac{s_{n}}{b} = q_{n} ; r_{n}
\label{eq:10-b}
\end{equation}
\subsection{Berechnungsablauf}
Nach dem ersten Rechenschritt (wenn der Quotient $q \neq 0$ ist), $n \geq 1$ gilt: 
\begin{equation}
s_{n}=q_{n-1}
\end{equation}
Das Gesamtergebnis $g$ ergibt sich wie folgt, wenn die Zielbasis $b < 10$ ist:
\begin{equation}
g = r_{0} * 10^0 + r_{1} * 10^1 \dots + r_{n} * 10^n
\label{eq:10-b_g9}
\end{equation}
Wenn die Zielbasis $b > 10$ ist, müssen die einzelnen Zeichendarstellungen der Reste $r_{n}$ rückwärts in eine Zeichenfolge zusammengesetzt werden. Das Ganze ist auch im Struktogramm in Abbildung~\ref{fig:struct2} dargestellt.
\begin{figure}
	\begin{struktogramm}(135,200)	
	\assign{Umwandlung von Basis 10 in Basis $b$.}
	\assign%
	{
		\begin{declaration}[Parameter:]
			\description{\pVar{b}}{Eine \pKey{int} Variable, die die zu benutzende Zahlenbasis angibt.}
			\description{\pVar{z}}{Eine \pKey{int} Variable, die die zu übersetzende Zahl der Zahlenbasis 10 enthält.}	
		\end{declaration}
		\begin{declaration}[lokale Variablen:]
			\description{\pVar{step}}{Eine \pKey{int} Variable, die den aktuellen Schritt anzeigt.}
			\description{\pVar{result}}{Eine \pKey{string} Variable, die das Endergebnis beinhaltet.}
			\description{\pVar{q}}{Eine \pKey{int} Variable für den berechneten Quotienten.}
			\description{\pVar{rest}}{Ein \pKey{int[]} Array zum Speichern der berechneten Restwerte.}
			\description{\pVar{i}}{Eine \pKey{int} Variable für zusätzliche Zählungen.}
			\description{\pVar{tmp}}{Eine \pKey{int} Variable für Zwischenergebnisse.}
			\description{\pVar{tmpS}}{Eine \pKey{string} Variable für Zwischenergebnisse.}
		\end{declaration}
	}
	\assign{\pVar{tmp} = 0}
	\assign{\pVar{step} = 0}
	\assign{\pVar{q}=\pVar{z}}
	\assign{\pVar{rest}-Array anlegen}
	\while[8]{\pVar{q}$ \neq 0$}
		\assign{\pVar{rest[step]} = \pVar{q} mod \pVar{b}}
		\assign{\pVar{q} = \pVar{q} / \pVar{b}}
		\assign{\pVar{step} = \pVar{step} + 1}
	\whileend 
	\ifthenelse{5}{5}
		{\pVar{b}$ > $ 10}{\sTrue}{\sFalse}
		\assign{\pVar{i}=0}
		\while[5]{\pVar{i}$ < $\pVar{step}}
			\sub{\pKey{char} für Wert \pVar{rest[i]} erhalten}
			\return{\pVar{tmpS} zurückgeben}
			\assign{\pVar{result} = \pVar{result} + \pVar{tmpS}}
			\assign{\pVar{i} = \pVar{i} + 1}
		\whileend
		\assign{\pVar{result} umdrehen}
	\change
		\assign{\pVar{i}=0}
		\while[5]{\pVar{i}$ < $\pVar{step}}
			\assign{\pVar{tmp} = \pVar{tmp} + \pVar{rest[i]}$ * 10^{i}$}
			\assign[21]{\pVar{i} = \pVar{i} + 1}
		\whileend
		\assign{\pVar{result} = \pVar{tmp} als \pKey{String}}
	\ifend
	\assign{\pVar{result} zurückgeben}
\end{struktogramm}
	\caption{Struktogramm Umwandlung vom Dezimalsystem}
	\label{fig:struct2}	
\end{figure}
Die Umwandlung einer Zahl $ > 9$ läuft ähnlich wie im Struktogramm in Abbildung~\ref{fig:struct1b}; nur umgekehrt. Siehe dazu das Struktogramm in Abbildung~\ref{fig:struct2b}
\begin{figure}
	\begin{struktogramm}(135,62)
	\assign{Berechnung eines Character-Werts}
	\assign%
	{
		\begin{declaration}[Parameter:]
			\description{\pVar{x}}{Eine \pKey{int} Variable, die den auszuwertenden wert angibt.}
		\end{declaration}
		\begin{declaration}[lokale Variablen:]
			\description{\pVar{result}}{Eine \pKey{char} Variable, die das Endergebnis beinhaltet.}
			\description{\pVar{z}}{Eine \pKey{int} Variable als Zwischenwert.}
		\end{declaration}
	}
	\assign{Lese den \texttt{ASCII} oder \texttt{UTF-8} Wert des Buchstabens \textbf{A} aus und schreibe es in das Zwischenergebnis \pVar{z}.} 
	\assign{\pVar{result}=\pVar{x} - 9 +  \pVar{z}}
	\assign{\pVar{result} zurückgeben}
\end{struktogramm}
	\caption{Struktogramm Umwandlung einer Zahl in einen Character-Wert}
	\label{fig:struct2b}
\end{figure}


\appendix
\chapter{Beispiele}
\section{Umwandlung ins Dezimalsystem}
Beispiele der Umrechnung von der Zahlenbasis $b_{1}=x$ in die Zahlenbasis $b_{2}=10$.
\subsection{Beispiel der Zahlenbasis $b_{1}=2$}
Im Binärsystem gibt es die zwei Zeichen 0 und 1. 
\subsubsection{Beispiel 1}
Die Binärzahl $2x100$ wird wie folgt nach Formel~\ref{eq:b-10} umgerechnet: von rechts nach links:
\begin{enumerate}
    \item An Stelle $s=1$: \[x_{1} = 2^{0} * 0 = 1 * 0 = 0\]
    \item An Stelle $s=2$: \[x_{2} = 2^{1} * 0 = 2 * 0 = 0\]
    \item An Stelle $s=3$: \[x_{3} = 2^{2} * 1 = 4 * 1 = 4\]
\end{enumerate}
Die Summierung von $x_{1}$ bis $x_{3}$ ist:
\[0 + 0 + 4 = 4\]
\subsubsection{Beispiel 2}
Die Binärzahl $2x110101$ wird wie folgt nach Formel~\ref{eq:b-10} umgerechnet: von rechts nach links:
\begin{enumerate}
    \item An Stelle $s=1$: \[x_{1} = 2^{0} * 1 = 1 * 1 = 1\]
    \item An Stelle $s=2$: \[x_{2} = 2^{1} * 0 = 2 * 0 = 0\]
    \item An Stelle $s=3$: \[x_{3} = 2^{2} * 1 = 4 * 1 = 4\]
    \item An Stelle $s=4$: \[x_{4} = 2^{3} * 0 = 8 * 0 = 0\]
    \item An Stelle $s=5$: \[x_{5} = 2^{4} * 1 = 16 * 1 = 16\]
    \item An Stelle $s=6$: \[x_{6} = 2^{5} * 1 = 32 * 1 = 32\]. 
\end{enumerate}
Die Summierung von $x_{1}$ bis $x_{6}$ ist:
\[1 + 0 + 4 + 0 + 16 + 32 = 53\]
\subsection{Beispiel der Zahlenbasis $b_{1}=8$}
Im Oktalsystem gibt es acht Zeichen von 0 bis 7. 
\subsubsection{Beispiel 1}
Die Oktalzahl $8x70$ wird wie folgt nach Formel~\ref{eq:b-10} umgerechnet: von rechts nach links:
\begin{enumerate}
    \item An Stelle $s=1$: \[x_{1} = 8^{0} * 0 = 1 * 0 = 0\]
    \item An Stelle $s=2$: \[x_{2} = 8^{1} * 7 = 8 * 7 = 56\]
\end{enumerate}
Die Summierung von $x_{1}$ bis $x_{2}$ ist:
\[0 + 56 = 56\]
\subsection{Beispiel der Zahlenbasis $b_{1}=16$}
Im Hexadezimalsystem gibt es sechzehn Zeichen von 0 bis $F$.
\subsubsection{Beispiel 1}
Die Hexadezimalzahl $16xD4$ wird wie folgt nach Formel~\ref{eq:b-10} von rechts nach links umgerechnet:
\begin{enumerate}
    \item An Stelle $s=1$: \[x_{1} = 16^{0} * 4 = 1 * 4 = 4\]
    \item An Stelle $s=2$: \[x_{2} = 16^{1} * (4 + 9) = 16 * 13 = 208\]
\end{enumerate}
Die Summierung von $x_{1}$ bis $x_{2}$ ist:
\[4 + 208 = 212\]
\subsubsection{Beispiel 2}
Die Hexadezimalzahl\index{hexadezimal} $16xAFFE$ wird wie folgt nach Formel~\ref{eq:b-10} von rechts nach links umgerechnet:
\begin{enumerate}
    \item An Stelle $s=1$: \[x_{1} = 16^{0} * (5 + 9) = 1 * 14 = 14\]
    \item An Stelle $s=2$: \[x_{2} = 16^{1} * (6 + 9) = 16 * 15 = 240\]
    \item An Stelle $s=3$: \[x_{3} = 16^{2} * (6 + 9) = 256 * 15 = 3840\]
    \item An Stelle $s=4$: \[x_{4} = 16^{3} * (1 + 9) = 4096 * 10 = 40960\]
\end{enumerate}
Die Summierung von $x_{1}$ bis $x_{4}$ ist:
\[14 + 240 + 3840 + 40960 = 45054\]

\section{Umrechnung vom Dezimalsystem in andere Zahlensysteme}
Beispiele der Umrechnung von der Zahlenbasis $b_{1}=10$ in die Zahlenbasis $b_{2}=x$.
\subsection{Beispiel der Zahlenbasis $b_{2}=2$}
Die Zahlenbasis\index{Zahlen!-basis} nennt sich \textbf{Binär}\index{binär}.
\subsubsection{Beispiel 1}
Die Dezimalzahl $z=13$ wird wie folgt nach Formel~\ref{eq:10-b} umgerechnet:
\begin{enumerate}
    \item An Stelle 1: $n=0$: \[\frac{s_{0}=z}{2}=\frac{13}{2}=q_{0}=6; r_{0}=1\]
    \item An Stelle 2: $n=1$: \[\frac{s_{1}=q_{0}}{2}=\frac{6}{2}=q_{1}=3; r_{1}=0\]
    \item An Stelle 3: $n=2$: \[\frac{s_{2}=q_{1}}{2}=\frac{3}{2}=q_{2}=1; r_{2}=1\]
    \item An Stelle 4: $n=3$: \[\frac{s_{3}=q_{2}}{2}=\frac{1}{2}=q_{3}=0; r_{3}=1\]
    \item Gesamtergebnis $g$ der Basis $b=2$: \[g=r_{0}*10^{0}+r_{1}*10^{1}+r_{2}*10^{2}+r_{3}*10^{3}\] \[g=1*1+0*10+1*100+1*1000=1101\]
\end{enumerate}
\(10x13=2x1101\)
\subsubsection{Beispiel 2}
Die Dezimalzahl $z=141$ wird wie folgt nach Formel~\ref{eq:10-b} umgerechnet:
\begin{enumerate}
    \item An Stelle 1: $n=0$: \[\frac{s_{0}=z}{2}=\frac{141}{2}=q_{0}=70; r_{0}=1\]
    \item An Stelle 2: $n=1$: \[\frac{s_{1}=q_{0}}{2}=\frac{70}{2}=q_{1}=35; r_{1}=0\]
    \item An Stelle 3: $n=2$: \[\frac{s_{2}=q_{1}}{2}=\frac{35}{2}=q_{2}=17; r_{2}=1\]
    \item An Stelle 4: $n=3$: \[\frac{s_{3}=q_{2}}{2}=\frac{17}{2}=q_{3}=8; r_{3}=1\]
    \item An Stelle 5: $n=4$: \[\frac{s_{4}=q_{3}}{2}=\frac{8}{2}=q_{4}=4; r_{4}=0\]
    \item An Stelle 6: $n=5$: \[\frac{s_{5}=q_{4}}{2}=\frac{4}{2}=q_{5}=2; r_{5}=0\]
    \item An Stelle 7: $n=6$: \[\frac{s_{6}=q_{5}}{2}=\frac{2}{2}=q_{6}=1; r_{5}=0\]
    \item An Stelle 8: $n=7$: \[\frac{s_{6}=q_{6}}{2}=\frac{1}{2}=q_{6}=0; r_{6}=1\]
    \item Gesamtergebnis $g$ der Basis $b=2$: \[g=r_{0}*10^{0}+r_{1}*10^{1}+r_{2}*10^{2}+r_{3}*10^{3}+r_{4}*10^{4}\]\[+r_{5}*10^{5}+r_{6}*10^{6}\] \[g=1*1+0*10+1*100+1*1000+0*10000\]\[+0*100000+0*1000000+1*10000000\]\[=10001101\]
\end{enumerate}
\(10x141=2x10001101\)

\subsection{Beispiel der Zahlenbasis $b_{2}=3$}
Die Zahlenbasis\index{Zahlen!-basis} nennt sich \textbf{Ternär}\index{ternär}.
\subsubsection{Beispiel 1}
Die Dezimalzahl $z=13$ wird wie folgt nach Formel~\ref{eq:10-b} umgerechnet:
\begin{enumerate}
	\item An Stelle 1: $n=0$: \[\frac{s_{0}=z}{3}=\frac{13}{3}=q_{0}=4;r_{0}=1\]
	\item An Stelle 2: $n=1$: \[\frac{s_{1}=q_{0}}{3}=\frac{4}{3}=q_{1}=1;r_{1}=1\]
	\item An Stelle 3: $n=2$: \[\frac{s_{2}=q_{1}}{3}=\frac{1}{3}=q_{2}=0;r_{2}=1\]
	\item Gesamtergebnis $g$ der Basis $b=3$: \[g=r_{0}*10^{0}+r_{1}*10^{1}+r_{2}*10^{2}\] \[g=1*1+1*10+1*100=111\]
\end{enumerate}
\(10x13=3x111\)
\subsubsection{Beispiel 2}
Die Dezimalzahl $z=141$ wird wie folgt nach Formel~\ref{eq:10-b} umgerechnet:
\begin{enumerate}
    \item An Stelle 1: $n=0$: \[\frac{s_{0}=z}{3}=\frac{141}{3}=q_{0}=47; r_{0}=0\]
    \item An Stelle 2: $n=1$: \[\frac{s_{1}=q_{0}}{3}=\frac{47}{3}=q_{1}=15; r_{1}=2\]
    \item An Stelle 3: $n=2$: \[\frac{s_{2}=q_{1}}{3}=\frac{15}{3}=q_{2}=5; r_{2}=0\]
    \item An Stelle 4: $n=3$: \[\frac{s_{3}=q_{2}}{3}=\frac{5}{3}=q_{3}=1; r_{3}=2\]
    \item An Stelle 5: $n=4$: \[\frac{s_{4}=q_{3}}{3}=\frac{1}{3}=q_{4}=0; r_{4}=1\]
    \item Gesamtergebnis $g$ der Basis $b=3$: \[g=r_{0}*10^{0}+r_{1}*10^{1}+r_{2}*10^{2}+r_{3}*10^{3}+r_{4}*10^{4}\] \[g=0*1+2*10+0*100+2*1000+1*10000\]\[=12020\]
\end{enumerate}
\(10x141=3x12020\)

\subsection{Beispiel der Zahlenbasis $b_{2}=8$}
Die Zahlenbasis\index{Zahlen!-basis} nennt sich \textbf{Oktal}\index{octal}.
\subsubsection{Beispiel 1}
Die Dezimalzahl $z=13$ wird wie folgt nach Formel~\ref{eq:10-b} umgerechnet:
\begin{enumerate}
	\item An Stelle 1: $n=0$: \[\frac{s_{0}=z}{8}=\frac{13}{8}=q_{0}=1;r_{0}=5\]
	\item An Stelle 2: $n=1$: \[\frac{s_{1}=q_{0}}{8}=\frac{1}{8}=q_{1}=0;r_{1}=1\]
	\item Gesamtergebnis $g$ der Basis $b=8$: \[g=r_{0}*10^{0}+r_{1}*10^{1}\] \[g=5*1+1*10=15\]
\end{enumerate}
\(10x13=8x15\)
\subsubsection{Beispiel 2}
Die Dezimalzahl $z=141$ wird wie folgt nach Formel~\ref{eq:10-b} umgerechnet:
\begin{enumerate}
    \item An Stelle 1: $n=0$: \[\frac{s_{0}=z}{8}=\frac{141}{8}=q_{0}=17; r_{0}=5\]
    \item An Stelle 2: $n=1$: \[\frac{s_{1}=q_{0}}{8}=\frac{17}{8}=q_{1}=2; r_{1}=1\]
    \item An Stelle 3: $n=2$: \[\frac{s_{2}=q_{1}}{8}=\frac{2}{8}=q_{2}=0; r_{2}=2\]
    \item Gesamtergebnis $g$ der Basis $b=8$: \[g=r_{0}*10^{0}+r_{1}*10^{1}+r_{2}*10^{2}\] \[g=5*1+1*10+2*100\]\[=215\]
\end{enumerate}
\(10x141=8x215\)

\chapter{Umsetzung in Programmiersprachen}
\section{JAVA-Codierung}
\lstset{language=JAVA, tabsize=3, caption={JAVA implementierung der Zahlenbasis}}
\label{listing:JAVA}
\begin{lstlisting}
public class Zahlenbase
{
}	
\end{lstlisting}
% \subsection{Umwandlung ins Dezimalsystem}
% \subsection{Umwandlung vom Dezimalsystem}
\section{PHP-Codierung}
Angesehen werden kann die Umsetzung in PHP\index{PHP}~5.x\index{PHP!PHP 5.x} unter 
"`\href{http://demo.hermann-bsd.de/zahlensysteme/}{http://demo.hermann-bsd.de/zahlensysteme}"'\footnote{Einiges funktioniert da noch nicht...}
\subsection{Interface}\index{Zahlen!-basen}
Zuerst als abstraktes Interface fuer die Definition von Zahlenbasen
\lstset{language=PHP, tabsize=3, caption={PHP Interface der Zahlenbasis}}
\label{listing:PHP:Interface}
\begin{lstlisting}
namespace ahbsd\Zahlensysteme
{
	/**
	 * Interface fuer grundlegende Funktionen, der Basis eines 
	 * Zahlensystems.
	 * 
	 * @author A. Hermann
	 * @copy Copyright &copy; 2016 
	 * Alexander Hermann - Beratung, Software, Design
	 * Zahlensysteme
	 *
	 * @version 1.0
	 */
	interface IBase
	{
		/**
		 * Gibt die Bezeichnung zurueck.
		 * @return string
		 */
		function GetName();
		
		/**
		 * Gibt das Zahlensystem als Integer zurueck.
		 * @return int Zahlensystem-Basis
		 */
		function GetSystem();
		
		/**
		 * Gibt das hoechstmoegliche Zeichen zurueck.
		 * @return char hoechstmoegliches Zeichen
		 */
		function GetMaxSign();
		
		/**
		 * Gibt das Zeichen der Basis fuer den Wert $x zurueck.
		 * @param int $x Wert x
		 * @return char Zeichen der Basis fuer den Wert x
		 */
		function GetSign($x);
	}
}
\end{lstlisting}
\subsection{Implementierung des Interfaces:}
\lstset{caption={PHP Implementierung der Zahlenbasis}}
\label{listing:PHP:Implementation}
\begin{lstlisting}
namespace ahbsd\Zahlensysteme
{
  /**
   * Basis eines Zahlensystems.
   * 
   * @author A. Hermann
   * @copy Copyright &copy; 2016 Alexander Hermann - Beratung, Software, Design
   * Zahlensysteme
   *
   * @version 1.0
   *
   */
  class Base implements IBase
  {
  	/**
  	 * Konstante, die die ASCII (und UTF-8) Position von 'A' speichert. 
  	 * @var int
  	 */
  	const A_POS_UTF8 = 65;
  	
  	/**
  	 * System-Name
  	 * @var string
  	 */
    private $systemName;
    
    /**
     * System als Integer-Zahl. Maximale Anzahl an Zeichen.
     * 
     * @var int
     */
    private $systemInt;
    
    /**
     * Hoechstm	gliches Zeichen.
     * 
     * @var char
     */
    private $maxSign;
    
    /**
     * Konstruktor
     * 
     * @param int $system Zahlensystem-Basis (Maximale Anzahl an Zeichen)
     * @param string $name (Optional) Bezeichnung des Zahlensystems
     */
    public function __construct($system, $name="")
    {
      $this->systemInt=intval($system);
      $this->systemName=$name;
      
      if ($name=="")
      {
        $this->systemName = sprintf("Basis %1\$s", intval($system));
      }
      $tmp = A_POS_UTF8 - 11 + intval($system);
      
      if ($system <= 10)
      {
        $this->maxSign = $system - 1;
      }
      else
      {
        $this->maxSign = mb_convert_encoding('&#' . $tmp . ';', 'UTF-8', 'HTML-ENTITIES');
      }
    }
    
    /**
     * (non-PHPdoc)
     * @see \ahbsd\Zahlensysteme\IBase::GetSign()
     */
    public function GetSign($x)
    {
      $tmp = 65-11+intval($x+1);
      $result = $x;
      
      if(intval($x) >= 10 || intval($x) < 0)
      {
        $result = mb_convert_encoding('&#' . $tmp . ';', 'UTF-8', 'HTML-ENTITIES');
      }
      
      return $result;
    }

    /**
     * (non-PHPdoc)
     * @see \ahbsd\Zahlensysteme\IBase::GetName()
     */
    public function GetName()
    {
      return $this->systemName;
    }
    
    /**
     * (non-PHPdoc)
     * @see \ahbsd\Zahlensysteme\IBase::GetSystem()
     */
    public function GetSystem()
    {
      return $this->systemInt;
    }
    
    /**
     * (non-PHPdoc)
     * @see \ahbsd\Zahlensysteme\IBase::GetMaxSign()
     */
    public function GetMaxSign()
    {
      return $this->maxSign;
    }
    
    /**
     * Statische Funktion zur Umwandlung einer Zahl aus dem Dezimalsystem in eine
     * Zahl des Zahlensystems bX.
     * 
     * @param int $b10 Umzuwandelnde Zahl aus dem Dezimalsystem.
     * @param int $bX Zahlensystem in das b10 umgewandelt werden soll.
     * @param bool $rechenweg (Optional) Gibt an, ob der Rechenweg angezeigt werden soll oder nicht; ohne Angabe standardmaessig FALSE. 
     * @return string Ergebnis in Basis bX
     */
    public static function Base10toBaseX($b10, $bX, $rechenweg=false)
    {
      $targetBase = new Base(intval($bX));
      $result = array();
      $restArray = array();
      $rOut = "";
      
      $quotient = intval($b10);
      $rest = 0;
      $cnt = 0;
      
      if ($rechenweg) 
      {
        echo "\n<!-- start Rechenweg --><pre>\n";
        echo "Rechenweg:\n";
      }
      
      while (intval($quotient) != 0)
      {
        $rest = $quotient % $bX;
        if ($rechenweg) echo "$quotient : $bX = " . intval($quotient / $bX) . " Rest $rest [" . $targetBase->GetSign($rest) . "]\n";
        $restArray[] = $rest;
        $quotient = intval($quotient / $bX);
      }
      if ($rechenweg) echo "--------------------\n";
      $cnt = count($restArray);
      
      for($i=0;$i < $cnt; $i++)
      {
        $result[$cnt - ($i + 1)] = $targetBase->GetSign($restArray[$i]);
      }  
      
      for($i=0; $i < $cnt; $i++)
      {
        $rOut .= $result[$i];
      }
      
      if ($rechenweg)
      {
        printf("Das Ergebnis der Umwandlung von %1\$s der Basis 10 in die %2\$s ist '%3\$s'\n", $b10, $targetBase->GetName(), $rOut);
        echo "</pre><!-- ende Rechenweg -->\n\n";
      }
      return $rOut;
    }
    
    /**
     * 
     * @param string $bxVal Der Wert der Basis $bX
     * @param int $bX Die Quell Basis.
     * @param bool $rechenweg Gibt an, ob der Rechenweg ausgegeben werden soll, 
     * 	oder nicht.
     * @return int Der Wert $bxVal umgerechnet in Basis 10.
     */
    public static function BaseXtoBase10($bxVal, $bX, $rechenweg=false)
    {
    	$sourceBase = new Base(intval($bX));
    	$step = strlen($bxVal) - 1;
    	$result = 0;
    	$z = 0;
    	$curCarCorrect = false;
    	$intW = 0;
    	
    	if ($rechenweg)
    	{
    		echo "\n<!-- start Rechenweg --><pre>\n";
    		echo "Rechenweg:\n";
    	}
    	
    	while ($step >= 0) {
    		$charW = $bxVal[$step];
    		$tmpIntVal = intval($charW, 10);
    		
    		$curCarCorrect = ('' . $tmpIntVal . ''==$charW); 
    		    		
    		if ($rechenweg) printf("%3\$s) Zeichen '%1\$s' an Stelle %2\$s ", $charW, $step, $z+1); 
    		
    		if ($curCarCorrect) {
    			$intW = $tmpIntVal;
    		}
    		else {
    			// CharWert Umrechnung
    			$tmp2 = ord($charW);
    			    			
    			$intW = intval($tmp2) - 65 + 10; // A_POS_UTF8;
    			
    			if ($rechenweg) printf("= (int) %1\$s", $intW);
    		}
    		
    		if ($rechenweg) echo "\n";
    		
    		$tmpR = 1;
    		
    		for ($i = 0; $i < $z; $i++) {
    			$tmpR = $tmpR * $bX;
    		}
    		
    		if ($rechenweg) printf("%1\$s^%2\$s=%3\$s\n%3\$s * %4\$s = ", $bX, $z, $tmpR, $intW);
    		
    		$tmpR = $tmpR * $intW;
    		
    		if ($rechenweg) echo $tmpR . "\n\n";
    		
    		$step--;
    		$result += $tmpR;
    		$z++;
    	}
    	
    	if ($rechenweg)
    	{
    		printf("Das Ergebnis der Umwandlung von '%1\$s' der %2\$s in die Basis 10 ist %3\$s\n", $bxVal, $sourceBase->GetName(), $result);
    		echo "</pre><!-- ende Rechenweg -->\n\n";
    	}
    	
    	return $result;
    }
  }
}
\end{lstlisting}

\section{C\#-Codierung}
\subsection{Interface}\index{Interface}
Zuerst ein generalisiertes Interface, um die grundlegenden Eigenschaften und Methoden zu definieren.
\lstloadlanguages{[Sharp]C}
\lstset{language=[Sharp]C, tabsize=3, caption={C-Sharp Interface der Zahlenbasis}}
\label{listing:C_sharp:interface}
\begin{lstlisting}
using System;

namespace AHBSD.Zahlensysteme
{
   public interface IBase
   {
      string Name { get; }
      uint System { get; }
      Char MaxSign { get; }
      Char GetSign(uint number);
      uint GetNumber(char sign);
   }
}
\end{lstlisting}
\subsection{Exception}\index{Exception}
Dann eine spezialisierte Exception, falls hier irgendetwas grandios daneben geht\footnote{z.B.: wenn jemand versucht eine Zahlenbasis unter 2 anzulegen\dots}
\lstset{caption={C-Sharp ZahlensystemException}}
\label{listing:C_sharp:Exception}
\begin{lstlisting}
using System;

namespace AHBSD.Zahlensysteme
{
   public class ZahlensystemException : Exception
   {
      private uint system;
      private IBase tryBase;
      
      public ZahlensystemException(IBase tb)
         : base(String.Format("Das kleinstmoegliche Zahlensystem ist 2; {0} ist zu klein!!", tb.System))
      {
         this.tryBase = tb;
         this.system = tb.System;
      }
      
      public ZahlensystemException(uint s)
         : base(String.Format("Das kleinstmoegliche Zahlensystem ist 2; {0} ist zu klein!!", s))
      {
         this.tryBase = null;
         this.system = s;
      }
      
      public ZahlensystemException(IBase tb, uint s)
         : base(String.Format("{0} {1} ist ausserhalb der Basis {2}.", tb, s, tb.System))
      {
         this.tryBase = tb;
         this.system = tb.System;
      }  
   }
}	
\end{lstlisting}

\subsection{Implementierung des Interfaces}\index{Interface!implementierung}
\lstset{caption={C-Sharp Implementierung des Interfaces}}
\label{listing:C_sharp:Implementierung}
\begin{lstlisting}
using System;
using System.Text;
using System.Collections.Generic;

namespace AHBSD.Zahlensysteme
{
   public class Base : IBase
   {
      private string name;
      private uint system;
      private Char maxSign;
      protected const uint A_POS=65;
      protected const uint ZERO_POS=48;
      
      public Base()
      {
         this.system = 10;
         this.SetName();
         this.maxSign = this.GetSign(9);
      }
      
      public Base(uint System)
      {
         this.system = System;
         this.SetName();
         this.maxSign = this.GetSign(System-1);
         
         if (System < 2)
         {
            throw new ZahlensystemException(this);
         }
      }
      
      public Base(uint System, string Name)
      {
         this.system = System;
         this.name = "Basis " + Name;
         this.maxSign = this.GetSign(System-1);
         
         if (System < 2)
         {
            throw new ZahlensystemException(this);
         }
      }
      
      protected void SetName()
      {
         this.name = "Basis " + this.system.ToString();
      }
      
#region IBase Members
      public string Name { get{ return this.name; } }
      public uint System { get { return this.system; } }
      public char MaxSign { get { return this.maxSign; } }
      public char GetSign(uint number)
      {
         char result;
         
         if (number < this.system)
         {
            result = GetSignByNumber(number);
         }
         else
         {
            throw new ZahlensystemException(this, number);
         }
         
         return result;
      }
      
      public uint GetNumber(char sign)
      {
          uint result = GetNumberBySign(sign);
          
          if (result >= this.system)
          {
             throw new ZahlensystemException(this, result);
          }
          
          return result;
      }
#endregion

      public static char GetSignByNumber(uint number)
      {
         char result = ' ';
         uint tmpI;
         
         if (number < 10)
         {
            tmpI = number + ZERO_POS;
         }
         else
         {
            tmpI = number - 10 + A_POS;
         }
         result = (char)tmpI;
         
         return result;
      }
      
      public static uint GetNumberBySign(char sign)
      {
         uint result = 0;
         uint tmp;
         
         if (Char.IsDigit(sign))
            {
               result = uint.Parse(sign.ToString());
            }
            else
            {
               tmp = (uint)sign;
               result = tmp - A_POS + 10;
            }
         
         return result;
      }
      
      public override string ToString()
      {
         StringBuilder result = new StringBuilder(this.name);
         result.Append("; MaxSign: ");
         result.Append(this.maxSign);
         
         return result.ToString();
      }
      
      public static List<string> Base10toBaseX(ulong valB10, uint X, bool Rechenweg)
      {
         List<string> result;
         string resultVal = String.Empty;
         IBase targetBase = new Base(X);
         ulong quotient = valB10;
         ulong tmpQuotient;
         uint rest = 0;
         List<uint> restList = new List<uint>();
         string[] fmt = new string[2];
         string tmpFmt;
         
         fmt[0] = "{0} : {1} = {2} Rest {3}";
         fmt[1] = "{0} : {1} = {2} Rest {3} [{4}]";
         
         if (!Rechenweg)
         {
            result = new List<string>(1);
            result.Add(String.Empty);
         }
         else
         {
            result = new List<string>();
            result.Add(String.Empty);
            result.Add("Rechenweg:");
         }
         
         while (quotient > 0)
         {
            tmpQuotient = quotient;
            rest = (uint)(quotient % X);
            quotient = quotient / X;
            restList.Add(rest);
            if (Rechenweg)
            {
               if (rest > 9)
               {
                  tmpFmt = fmt[1];
               }
               else
               {
                  tmpFmt = fmt[0];
               }
               result.Add(String.Format(tmpFmt, tmpQuotient, X, quotient, rest, targetBase.GetSign((uint)rest)));
            }
            
            resultVal = targetBase.GetSign((uint)rest).ToString() + resultVal;
         }
         
         if (Rechenweg) result.Add(String.Format("Das Ergebnis der Umwandlung von {0} der Basis 10 in die Basis {1} ist '{2}'", valB10, X, resultVal));
         result[0] = resultVal;
         
         return result;
      }
      
      public static List<string> BaseXtoBase10(string valBX, uint X, bool Rechenweg)
      {
         List<string> result;
         IBase sourceSystem = new Base(X);
         
         int step = valBX.Length -1;
         char[] valBXC = valBX.ToCharArray();
         char charW;
         uint intW = 0;
         uint tmp;
         int zaehler = 0;
         string[] fmt = new string[2];
         string tmpFmt;
         
         fmt[0] = "Wert an Position {0}: '{1}' = {2} int; {3}^{0} * {2} = {4} * {2} = {5}";
         fmt[1] = "Wert an Position {0}: {1}; {3}^{0} * {2} = {4} * {2} = {5}";
         
         if (!Rechenweg)
         {
            result = new List<string>(1);
            result.Add(String.Empty);
         }
         else
         {
            result = new List<string>();
            result.Add(String.Empty);
            result.Add("Rechenweg:");
         }
         
         while (step >= 0)
         {
            charW = valBXC[step];
            
            tmp = sourceSystem.GetNumber(charW);
            
            if (Char.IsDigit(charW))
            {
               tmpFmt = fmt[1];
            }
            else
            {
               tmpFmt = fmt[0];
            }
            
            if (Rechenweg) result.Add(String.Format(tmpFmt, zaehler, charW, tmp, X, (uint)Math.Pow(X, zaehler), tmp * (uint)Math.Pow(X, zaehler)));
            intW += tmp * (uint)Math.Pow(X, zaehler);
            
            zaehler++;
            step--;
         }
         result[0] = intW.ToString();
         
         if (Rechenweg) result.Add(String.Format("Das Ergebnis der Umwandlung von '{0}' der Basis {1} in die Basis 10 ist {2}", valBX, X, result[0]));
         return result;
      }
      
      public static List<string> BaseXtoBaseY(string valX, uint X, uint Y, bool Rechenweg)
      {
         List<string> result;
         uint tmp;
         int l;
         string tmpS;
         
         if (X!=10)
         {
            result = Base.BaseXtoBase10(valX, X, Rechenweg);
            tmp = uint.Parse(result[0]);
            l = result.Count;
         }
         else
         {
            tmp = uint.Parse(valX);
            l = 0;
            result = new List<string>();
         }
         
         if (l > 0)
         {
            if (Y!=10)
            {
            result.AddRange(Base.Base10toBaseX(tmp, Y, Rechenweg));
            }
            else
            {
               result.Add(tmp.ToString());
            }
         }
         else
         {
            if (Y!=10)
            {
               result = Base.Base10toBaseX(tmp, Y, Rechenweg);
            }
            else
            {
               result = new List<string>();
               result.Add(tmp.ToString());
            }
         }
         tmpS = String.Format("Das Ergebnis der Umwandlung von '{0}' der Basis {1} in die Basis {3} ist '{2}'", valX, X, result[l], Y);
         if (Rechenweg) result.Add(tmpS);
         result.Add(result[l]);
         return result;
      }
   }
}
\end{lstlisting}
\bibliography{zahlenbasen}
\printindex
\end{document}
