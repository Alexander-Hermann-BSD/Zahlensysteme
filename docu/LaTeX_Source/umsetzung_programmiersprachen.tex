\chapter{Umsetzung in Programmiersprachen}
% \section{JAVA-Codierung}
% \subsection{Umwandlung ins Dezimalsystem}
% \subsection{Umwandlung vom Dezimalsystem}
\section{PHP-Codierung}
Angesehen werden kann die Umsetzung in PHP\index{PHP}~5.x\index{PHP!PHP 5.x} unter 

http://demo.hermann-bsd.de/zahlensysteme/\footnote{Einiges funktioniert da noch nicht...}
\subsubsection{Zuerst als abstraktes Interface fuer die Definition von Zahlenbasen}\index{Zahlen!-basen}
\lstset{language=PHP, tabsize=3, caption={PHP Interface der Zahlenbasis}}
\label{listing:PHP:Interface}
\begin{lstlisting}
namespace ahbsd\Zahlensysteme
{
	/**
	 * Interface fuer grundlegende Funktionen, der Basis eines 
	 * Zahlensystems.
	 * 
	 * @author A. Hermann
	 * @copy Copyright &copy; 2016 
	 * Alexander Hermann - Beratung, Software, Design
	 * Zahlensysteme
	 *
	 * @version 1.0
	 */
	interface IBase
	{
		/**
		 * Gibt die Bezeichnung zurueck.
		 * @return string
		 */
		function GetName();
		
		/**
		 * Gibt das Zahlensystem als Integer zurueck.
		 * @return int Zahlensystem-Basis
		 */
		function GetSystem();
		
		/**
		 * Gibt das hoechstmoegliche Zeichen zurueck.
		 * @return char hoechstmoegliches Zeichen
		 */
		function GetMaxSign();
		
		/**
		 * Gibt das Zeichen der Basis fuer den Wert $x zurueck.
		 * @param int $x Wert x
		 * @return char Zeichen der Basis fuer den Wert x
		 */
		function GetSign($x);
	}
}
\end{lstlisting}
\subsubsection{Implementierung des Interfaces:}
\lstset{caption={PHP Implementierung der Zahlenbasis}}
\label{listing:PHP:Implementation}
\begin{lstlisting}
namespace ahbsd\Zahlensysteme
{
  /**
   * Basis eines Zahlensystems.
   * 
   * @author A. Hermann
   * @copy Copyright &copy; 2016 Alexander Hermann - Beratung, Software, Design
   * Zahlensysteme
   *
   * @version 1.0
   *
   */
  class Base implements IBase
  {
  	/**
  	 * Konstante, die die ASCII (und UTF-8) Position von 'A' speichert. 
  	 * @var int
  	 */
  	const A_POS_UTF8 = 65;
  	
  	/**
  	 * System-Name
  	 * @var string
  	 */
    private $systemName;
    
    /**
     * System als Integer-Zahl. Maximale Anzahl an Zeichen.
     * 
     * @var int
     */
    private $systemInt;
    
    /**
     * Hoechstm	gliches Zeichen.
     * 
     * @var char
     */
    private $maxSign;
    
    /**
     * Konstruktor
     * 
     * @param int $system Zahlensystem-Basis (Maximale Anzahl an Zeichen)
     * @param string $name (Optional) Bezeichnung des Zahlensystems
     */
    public function __construct($system, $name="")
    {
      $this->systemInt=intval($system);
      $this->systemName=$name;
      
      if ($name=="")
      {
        $this->systemName = sprintf("Basis %1\$s", intval($system));
      }
      $tmp = A_POS_UTF8 - 11 + intval($system);
      
      if ($system <= 10)
      {
        $this->maxSign = $system - 1;
      }
      else
      {
        $this->maxSign = mb_convert_encoding('&#' . $tmp . ';', 'UTF-8', 'HTML-ENTITIES');
      }
    }
    
    /**
     * (non-PHPdoc)
     * @see \ahbsd\Zahlensysteme\IBase::GetSign()
     */
    public function GetSign($x)
    {
      $tmp = 65-11+intval($x+1);
      $result = $x;
      
      if(intval($x) >= 10 || intval($x) < 0)
      {
        $result = mb_convert_encoding('&#' . $tmp . ';', 'UTF-8', 'HTML-ENTITIES');
      }
      
      return $result;
    }

    /**
     * (non-PHPdoc)
     * @see \ahbsd\Zahlensysteme\IBase::GetName()
     */
    public function GetName()
    {
      return $this->systemName;
    }
    
    /**
     * (non-PHPdoc)
     * @see \ahbsd\Zahlensysteme\IBase::GetSystem()
     */
    public function GetSystem()
    {
      return $this->systemInt;
    }
    
    /**
     * (non-PHPdoc)
     * @see \ahbsd\Zahlensysteme\IBase::GetMaxSign()
     */
    public function GetMaxSign()
    {
      return $this->maxSign;
    }
    
    /**
     * Statische Funktion zur Umwandlung einer Zahl aus dem Dezimalsystem in eine
     * Zahl des Zahlensystems bX.
     * 
     * @param int $b10 Umzuwandelnde Zahl aus dem Dezimalsystem.
     * @param int $bX Zahlensystem in das b10 umgewandelt werden soll.
     * @param bool $rechenweg (Optional) Gibt an, ob der Rechenweg angezeigt werden soll oder nicht; ohne Angabe standardmaessig FALSE. 
     * @return string Ergebnis in Basis bX
     */
    public static function Base10toBaseX($b10, $bX, $rechenweg=false)
    {
      $targetBase = new Base(intval($bX));
      $result = array();
      $restArray = array();
      $rOut = "";
      
      $quotient = intval($b10);
      $rest = 0;
      $cnt = 0;
      
      if ($rechenweg) 
      {
        echo "\n<!-- start Rechenweg --><pre>\n";
        echo "Rechenweg:\n";
      }
      
      while (intval($quotient) != 0)
      {
        $rest = $quotient % $bX;
        if ($rechenweg) echo "$quotient : $bX = " . intval($quotient / $bX) . " Rest $rest [" . $targetBase->GetSign($rest) . "]\n";
        $restArray[] = $rest;
        $quotient = intval($quotient / $bX);
      }
      if ($rechenweg) echo "--------------------\n";
      $cnt = count($restArray);
      
      for($i=0;$i < $cnt; $i++)
      {
        $result[$cnt - ($i + 1)] = $targetBase->GetSign($restArray[$i]);
      }  
      
      for($i=0; $i < $cnt; $i++)
      {
        $rOut .= $result[$i];
      }
      
      if ($rechenweg)
      {
        printf("Das Ergebnis der Umwandlung von %1\$s der Basis 10 in die %2\$s ist '%3\$s'\n", $b10, $targetBase->GetName(), $rOut);
        echo "</pre><!-- ende Rechenweg -->\n\n";
      }
      return $rOut;
    }
    
    /**
     * 
     * @param string $bxVal Der Wert der Basis $bX
     * @param int $bX Die Quell Basis.
     * @param bool $rechenweg Gibt an, ob der Rechenweg ausgegeben werden soll, 
     * 	oder nicht.
     * @return int Der Wert $bxVal umgerechnet in Basis 10.
     */
    public static function BaseXtoBase10($bxVal, $bX, $rechenweg=false)
    {
    	$sourceBase = new Base(intval($bX));
    	$step = strlen($bxVal) - 1;
    	$result = 0;
    	$z = 0;
    	$curCarCorrect = false;
    	$intW = 0;
    	
    	if ($rechenweg)
    	{
    		echo "\n<!-- start Rechenweg --><pre>\n";
    		echo "Rechenweg:\n";
    	}
    	
    	while ($step >= 0) {
    		$charW = $bxVal[$step];
    		$tmpIntVal = intval($charW, 10);
    		
    		$curCarCorrect = ('' . $tmpIntVal . ''==$charW); 
    		    		
    		if ($rechenweg) printf("%3\$s) Zeichen '%1\$s' an Stelle %2\$s ", $charW, $step, $z+1); 
    		
    		if ($curCarCorrect) {
    			$intW = $tmpIntVal;
    		}
    		else {
    			// CharWert Umrechnung
    			$tmp2 = ord($charW);
    			    			
    			$intW = intval($tmp2) - 65 + 10; // A_POS_UTF8;
    			
    			if ($rechenweg) printf("= (int) %1\$s", $intW);
    		}
    		
    		if ($rechenweg) echo "\n";
    		
    		$tmpR = 1;
    		
    		for ($i = 0; $i < $z; $i++) {
    			$tmpR = $tmpR * $bX;
    		}
    		
    		if ($rechenweg) printf("%1\$s^%2\$s=%3\$s\n%3\$s * %4\$s = ", $bX, $z, $tmpR, $intW);
    		
    		$tmpR = $tmpR * $intW;
    		
    		if ($rechenweg) echo $tmpR . "\n\n";
    		
    		$step--;
    		$result += $tmpR;
    		$z++;
    	}
    	
    	if ($rechenweg)
    	{
    		printf("Das Ergebnis der Umwandlung von '%1\$s' der %2\$s in die Basis 10 ist %3\$s\n", $bxVal, $sourceBase->GetName(), $result);
    		echo "</pre><!-- ende Rechenweg -->\n\n";
    	}
    	
    	return $result;
    }
  }
}
\end{lstlisting}